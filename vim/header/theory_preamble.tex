%%%%%%%%%%%%%%%%%%%%%%%%%%%%%%%%%%%%%%%%%%%%%%%%%%%%%%%%%%%%%%%%%%%%%%%%%%%%%%%
%           ROESLER EPPS PREAMBLE
%%%%%%%%%%%%%%%%%%%%%%%%%%%%%%%%%%%%%%%%%%%%%%%%%%%%%%%%%%%%%%%%%%%%%%%%%%%%%%%
\usepackage{etex}
\usepackage[english]{babel}
\usepackage[T1]{fontenc}
\usepackage{amsmath, amssymb, latexsym, marvosym, mathrsfs}
\usepackage{ulem, ar, textcomp} % ar gives AR aspect ratio symbol, textcomp for \textquotedbl "
% \usepackage{natbib} 
\usepackage[sort,square,comma,numbers,compress]{natbib} 
% \usepackage[authoryear]{natbib} 
\usepackage{color, fancybox}
\usepackage{graphics, graphicx}
\usepackage{epsfig, epstopdf}
\usepackage{setspace}
\usepackage{longtable}
\usepackage{verbatim}
\usepackage{mathtools,empheq}  % for boxed equations:    \Aboxed{...}
% \usepackage[sectionbib]{chapterbib}   % use for multiple bibliographies
% \usepackage{layouts,layout,fancyhdr,multicol}
\usepackage[font=small,hypcap=true]{caption}                % link to top of figures and subfigures
\usepackage[font=small,hypcap=true,list=true]{subcaption}   % use for subfigures

%%%%%%%%%%%%%%%%%%%%%%%%%%%%%%%%%%%%%%%%%%%%%%%%%
\usepackage{hyperref}   % use hyperref alone to use default article format
% \usepackage[hidelinks]{hyperref}
% \hypersetup{
%     colorlinks=true, %set true if you want colored links
%     linktoc=all,     %set to all if you want both sections and subsections linked
%     citecolor=black,
%     filecolor=black,
%     linkcolor=black,
%     urlcolor=black
% }


%%%%%%%%%%%%%%%%%%%%%%%%%%%%%%%%%%%%%%%%%%%%%%%%%
%% NO HYPHENATION
\tolerance=1
\emergencystretch=\maxdimen
\hyphenpenalty=10000
\hbadness=10000




  
%%%%%%%%%%%%%%%%%%%%%%%%%%%%%%%%%%%%%%%%%%%%%%%%%
\newcommand{\mbf}{\mathbf}
\newcommand{\un}{\underline}  


%%%%%%%%%%%%%%%%%%%%%%%%%%%%%%%%%%%%%%%%%%%%%%%%%
%--------------  Close sqrt over contents
\makeatletter
\let\oldr@@t\r@@t
\def\r@@t#1#2{%
  \setbox0=\hbox{$\oldr@@t#1{#2\,}$}\dimen0=\ht0
  \advance\dimen0-0.2\ht0
  \setbox2=\hbox{\vrule height\ht0 depth -\dimen0}%
  {\box0\lower0.4pt\box2}}
\LetLtxMacro{\oldsqrt}{\sqrt}
\renewcommand*{\sqrt}[2][\ ]{\oldsqrt[#1]{#2}}
\makeatother

%%%%%%%%%%%%%%%%%%%%%%%%%%%%%%%%%%%%%%%%%%%%%%%%%
%-------------- code to create triple underline command
\makeatletter
\newcommand\uuuline{\bgroup\markoverwith%
%   {%
%     \textcolor{black}{\rule[-0.5ex]{2pt}{0.4pt}}%
%     \llap{\textcolor{blue}{\rule[-0.7ex]{2pt}{0.4pt}}}%
%     \llap{\textcolor{green}{\rule[-0.9ex]{2pt}{0.4pt}}}%
%   }%
   {%
          {\rule[-0.5ex]{2pt}{0.4pt}}%
     \llap{\rule[-0.7ex]{2pt}{0.4pt}}%
     \llap{\rule[-0.9ex]{2pt}{0.4pt}}%
   }%
   \ULon}
\makeatother



%%%%%%%%%%%%%%%%%%%%%%%%%%%%%%%%%%%%%%%%%%%%%%%%%
% define partial and variational
\newcommand{\p}{\partial}
\renewcommand{\d}{\delta}
\newcommand{\ddt}{\frac{d}{dt}}
\newcommand{\ppt}{\frac{\partial}{\partial t}}
\newcommand{\D}{{\scriptstyle \Delta}\hspace{-0.25 em}}

% Derivatives
\newcommand{\dd}{\mathrm{d}}
\newcommand{\ppx}[2]{\frac{\partial #1}{\partial #2}}
\newcommand{\psx}[2]{\frac{\partial^2 #1}{\partial #2^2}}
\newcommand{\ddx}[2]{\frac{\dd #1}{\dd #2}}
\newcommand{\dsx}[2]{\frac{\dd^2 #1}{\dd #2^2}}
\newcommand{\DDx}[2]{\frac{\mathrm{D}#1}{\mathrm{D} #2}}

% Use for fractions with large terms
\newcommand{\ddfrac}[2]{\frac{\displaystyle #1}{\displaystyle #2}}

% define sine and cosine
\renewcommand{\t}{\theta}
\newcommand{\s}{\sin\theta}
\renewcommand{\c}{\cos\theta}

%%%%%%%%%%%%%%%%%%%%%%%%%%%%%%%%%%%%%%%%%%%%%%%%%
% unit vectors:
\newcommand{\ex}{\textbf{e}_\textbf{x}}
\newcommand{\ey}{\textbf{e}_\textbf{y}}
\newcommand{\ez}{\textbf{e}_\textbf{z}}
\newcommand{\er}{\textbf{e}_\textbf{r}}
\newcommand{\et}{\textbf{e}_\mbf{\theta}}


\newcommand{\ntilde}{\mathbf{\tilde{n}}}
\newcommand{\chat}{\mathbf{\hat{c}}}
\newcommand{\shat}{\mathbf{\hat{s}}}
\newcommand{\nhat}{\mathbf{\hat{n}}}
\newcommand{\that}{\mathbf{\hat{t}}}
\newcommand{\ihat}{\mathbf{\hat{i}}}
\newcommand{\jhat}{\mathbf{\hat{j}}}
\newcommand{\khat}{\mathbf{\hat{k}}}
\newcommand{\xhat}{\mathbf{\hat{x}}}
\newcommand{\yhat}{\mathbf{\hat{y}}}
\newcommand{\zhat}{\mathbf{\hat{z}}}
\newcommand{\phat}{\mathbf{\hat{\phi}}}

%%%%%%%%%%%%%%%%%%%%%%%%%%%%%%%%%%%%%%%%%%%%%%%%%
% Miscellaneous
\newcommand{\gradv}{\uuline{\nabla\textbf{v}}}
\newcommand{\gradV}{\uuline{\nabla\textbf{V}}}
\renewcommand{\l}{\ell}
\renewcommand{\v}{\textbf{v}}
  \newcommand{\V}{\textbf{V}}
\newcommand{\w}{\omega}
\newcommand{\wn}{\omega_n}

\newcommand{\Ln}{\text{Ln}}
\newcommand{\Log}{\text{Log}}
\newcommand{\argz}{\text{\,arg\,}}
\newcommand{\cis}{\text{\,cis\,}}
\newcommand{\cist}{\cos\theta+i\sin\theta}

\newcommand{\TAU}{\mat{\tau}}
\newcommand{\SIGMA}{\mat{\sigma}}

\newcommand{\cc}[1]{\multicolumn{1}{c}{#1}}   % center one item in table
\newcommand{\tqd}{\textquotedbl} % needs \usepackage{textcomp}!!!
\newcommand{\ie}{{\it i.e.\ }}
\newcommand{\eg}{{\it e.g.\ }}
\renewcommand{\deg}{^\circ}

%%%%%%%%%%%%%%%%%%%%%%%%%%%%%%%%%%%%%%%%%%%%%%%%%
\let\real\Re
\let\imag\Im
  \newcommand{\De}{\mathit{De}}
\renewcommand{\Re}{\mathit{Re}}
  \newcommand{\Fr}{\mathit{Fr}}
  \newcommand{\Ma}{\mathit{Ma}}
  \newcommand{\Nu}{\mathit{Nu}}
\renewcommand{\Pr}{\mathit{Pr}}
  \newcommand{\St}{\mathit{St}}


%\newcommand{\vol}{\text{\sout{\ensuremath{V}}}}   % volume == strikethrough V 
 \newcommand{\vol}{{\cal V}}
 \newcommand{\CV}{{\cal CV}}
 \newcommand{\CS}{{\cal CS}} 
 \newcommand{\MV}{{\cal MV}}
 \newcommand{\MS}{{\cal MS}}
  % \newcommand{\Z}{{\cal Z}}
\renewcommand{\O}{{\cal O}}    % order of magnitude symbol
%\renewcommand{\O}{\ensuremath{\mathcal{O}}}  % order of magnitude symbol




%%%%%%%%%%%%%%%%%%%%%%%%%%%%%%%%%%%%%%%%%%%%%%%%%
\newcommand{\res}{{\cal R}}
\renewcommand{\L}{\left}
\newcommand{\R}{\right}
\renewcommand{\(}{\left(}
\renewcommand{\)}{\right)}
\renewcommand{\[}{\left[}
\renewcommand{\]}{\right]}
\newcommand{\half}{\tfrac{1}{2}}
\newcommand{\third}{\tfrac{1}{3}}
\newcommand{\quarter}{\tfrac{1}{4}}
\newcommand{\ra}{\rightarrow}
\newcommand{\alfa}{\alpha}

\newcommand{\T}{\intercal} % transpose symbol
\renewcommand{\th}{{\text{th}}}

%%%%%%%%%%%%%%%%%%%%%%%%%%%%%%%%%%%%%%%%%%%%%%%%%
% sums
\newcommand{\Isum}[1]{\sum\limits_{#1=0}^\infty}
\newcommand{\MIsum}[1]{\sum\limits_{#1=-\infty}^\infty}
\newcommand{\Nsum}[1]{\sum\limits_{#1=1}^N}
\newcommand{\Xsum}[2]{\sum\limits_{#1=1}^#2}

% integrals
\newcommand{\intc}{\int\limits_C}
\newcommand{\intcn}[1]{\int\limits_{C_#1}}
\newcommand{\ointc}{\oint\limits_C}

\newcommand{\intX}[1]{\int\limits_{#1}}

% %%%%%%%%%%%%%%%%%%%%%%%%%%%%%%%%%%%%%%%%%%%%%%%%%
% % indices
%   \newcommand{\m}{{\scriptstyle (m)}}
%   \newcommand{\n}{{\scriptstyle (n)}}
% \renewcommand{\i}{{\scriptstyle (i)}}
% \renewcommand{\j}{{\scriptstyle (j)}}
% \newcommand{\one}{{\scriptstyle (1)}}
% \newcommand{\I}{{\scriptstyle (I)}}
% \newcommand{\J}{{\scriptstyle (J)}}
% \newcommand{\Mp}{{\scriptstyle (M)}}
% \newcommand{\Np}{{\scriptstyle (N)}}
% \newcommand{\N}{{\scriptstyle (N)}}
%
% \newcommand{\mm}{{\scriptstyle (m,m)}}
% \newcommand{\nn}{{\scriptstyle (n,n)}}
% \newcommand{\ii}{{\scriptstyle (i,i)}}
% \renewcommand{\ij}{{\scriptstyle (i,j)}}
% \newcommand{\mi}{{\scriptstyle (m,i)}}
% \newcommand{\mj}{{\scriptstyle (m,j)}}
% \newcommand{\mn}{{\scriptstyle (m,n)}}
% \newcommand{\nm}{{\scriptstyle (n,m)}}
% \newcommand{\im}{{\scriptstyle (i,m)}}
%
% \newcommand{\ipo}{{\scriptstyle (i+1)}}
% \newcommand{\jpo}{{\scriptstyle (j+1)}}
% \newcommand{\mpo}{{\scriptstyle (m+1)}}
% \newcommand{\npo}{{\scriptstyle (n+1)}}
\newcommand{\imo}{{\scriptstyle (i-1)}}
% \newcommand{\jmo}{{\scriptstyle (j-1)}}
% \newcommand{\mmo}{{\scriptstyle (m-1)}}
% \newcommand{\nmo}{{\scriptstyle (n-1)}}
%
% \newcommand{\iip}{{\scriptstyle (i,i+1)}}
% \newcommand{\mip}{{\scriptstyle (m,i+1)}}
% \newcommand{\mnp}{{\scriptstyle (m,n+1)}}
%
% \newcommand{\Ij}{{\scriptstyle (I,j)}}
%
%%%%%%%%%%%%%%%%%%%%%%%%%%%%%%%%%%%%%%%%%%%%%%%%%

% From OMAE2014 PAPER
\renewcommand{\i}{{\scriptstyle (i)}}
\renewcommand{\j}{{\scriptstyle (j)}}
\renewcommand{\ij}{{\scriptstyle (i,j)}}
\newcommand{\Ij}{{\scriptstyle (I,j)}}
\newcommand{\I}{{\scriptstyle (I)}}
\newcommand{\J}{{\scriptstyle (J)}}
\newcommand{\N}{{\scriptstyle (N)}}
\newcommand{\ip}{{\scriptstyle (i+1)}}
\newcommand{\im}{{\scriptstyle (i-1)}}
\newcommand{\jm}{{\scriptstyle (j-1)}}

\newcommand{\CR}{\texttt{CyROD} }
\newcommand{\CRV}{\texttt{CyROD~1.18} }
\newcommand{\Matlab}{\textsc{Matlab} }

\newcommand{\Z}{\mathscr{Z}}
%%%%%%%%%%%%%%%%%%%%%%%%%%%%%%%%%%%%%%%%%%%%%%%%%

\newcommand{\norm}[1]{\lVert #1 \rVert}
\newcommand{\vect}[1]{\boldsymbol{#1}}
\newcommand{\code}{\texttt}
\newcommand{\GB}{\Gamma_b}
\newcommand{\GS}{\Gamma_s}
\newcommand{\GW}{\Gamma_w}
\newcommand{\bp}{\beta(\phi)}





%%%%%%%%%%%%%%%%%%%%%%%%%%%%%%%%%%%%%%%%%%%%%%%%%%
% This code enables subequations within the align environment
\usepackage{etoolbox}

% let \theparentequation use the same definition as equation
 \let\theparentequation\theequation

% change every occurence of "equation" to "parentequation"
\patchcmd{\theparentequation}{equation}{parentequation}{}{}

\renewenvironment{subequations}[1][]{%              optional argument: label-name for (first) parent equation
 \refstepcounter{equation}%
%  \def\theparentequation{\arabic{parentequation}}  % we patched it already :)
 \setcounter{parentequation}{\value{equation}}%    parentequation = equation
 \setcounter{equation}{0}%                         (sub)equation  = 0
 \def\theequation{\theparentequation\alph{equation}}% 
 \let\parentlabel\label%                           Evade sanitation performed by amsmath
 \ifx\\#1\\\relax\else\label{#1}\fi%               #1 given: \label{#1}, otherwise: nothing
 \ignorespaces
}{%
 \setcounter{equation}{\value{parentequation}}%    equation = subequation
 \ignorespacesafterend
}

\newcommand*{\nextParentEquation}[1][]{%            optional argument: label-name for (first) parent equation
 \refstepcounter{parentequation}%                  parentequation++
 \setcounter{equation}{0}%                         equation = 0
 \ifx\\#1\\\relax\else\parentlabel{#1}\fi%         #1 given: \label{#1}, otherwise: nothing
}

%% Minimal Working Example:
%%	\begin{equation}
%%	  0 \neq 1
%%	\end{equation}
%%	\begin{subequations}[eq:2]% or: \label{eq:2}
%%	\begin{align}%              or: \parentlabel{eq:2}
%%	    A & = B+1 \label{eq:2a} \\
%%	  B+1 & = C   \label{eq:2b}
%%	\intertext{therefore}\nextParentEquation[eq:3]% or: \nextParentEquation\parentlabel{eq:3}
%%	    A & = C   \label{eq:3a} \\
%%	    B & = C-1 \label{eq:3b}
%%	\end{align}
%%	\end{subequations}
%%	\eqref{eq:2}: \eqref{eq:2a}, \eqref{eq:2b} \\ \eqref{eq:3}: \eqref{eq:3a}, \eqref{eq:3b}
%%	\begin{equation}
%%	  1 \neq 2
%%	\end{equation}
%%===============================================================================
