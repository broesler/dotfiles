\usepackage[english]{babel}
\usepackage[letterpaper]{geometry}
\usepackage[T1]{fontenc}
\usepackage{graphicx}
\usepackage{epsfig, epstopdf}
\usepackage{amsmath, amssymb, latexsym, empheq, mathtools}
\usepackage{mathrsfs} % script font for fancy math
\usepackage{verbatim, url}
\usepackage{fancyhdr, listings, color}
\usepackage{parskip}
\usepackage{array, multirow}
\usepackage{hyperref}
\hypersetup{
    linktoc=all,    % link table of contents to sections
    colorlinks,
    allcolors=black }
\usepackage[font=small,hypcap=true]{caption}                % link to top of figures and subfigures
\usepackage[font=small,hypcap=true,list=true]{subcaption}   % use for subfigures

% Include pdf pages directly in document
% \usepackage{pdfpages}

% \usepackage{mathptmx}  % Times font
% \usepackage[arial-urw]{mathdesign}
% \usepackage{ebgaramond}
% \renewcommand{\familydefault}{ptm}    % Use times font

\DeclareGraphicsRule{.tif}{png}{.png}{`convert #1 `dirname #1`/`basename #1 .tif`.png}

% Define colors for code
\definecolor{dkgreen}{rgb}{0,0.6,0}
\definecolor{gray}{rgb}{0.5,0.5,0.5}
\definecolor{mauve}{rgb}{0.58,0,0.82}

\lstset{
  % language = Matlab,                      % the language of the code
  basicstyle = \scriptsize\ttfamily,    % the size of the fonts that are used for the code
  numbers = left,                         % where to put the line-numbers
  numberstyle = \tiny\color{gray},        % the style that is used for the line-numbers
  stepnumber = 1,                         % the step between two line-numbers.
  numbersep = 10pt,                        % how far the line-numbers are from the code
  breaklines = true,                      % sets automatic line breaking
  showspaces = false,                     % show spaces adding particular underscores
  showstringspaces = false,               % underline spaces within strings
  showtabs = false,                       % show tabs within strings adding particular underscores
  keywordstyle = \color{blue},            % keyword style
  commentstyle = \color{dkgreen},         % comment style
  stringstyle = \color{mauve}             % string literal style
}

\renewcommand{\baselinestretch}{1.15} 	% 1.15 spacing between lines

\setlength{\parindent}{20 pt}

% page layout
%\textwidth  390pt
\textheight 650pt

% paperheight
\voffset         0 in
\topmargin -0.5 in
%\headheight 0pt
\headsep 2 em
\footskip 36 pt
\raggedbottom


%% NO HYPHENATION
\tolerance=1
\emergencystretch=\maxdimen
\hyphenpenalty=10000
\hbadness=10000

% MAKE SQRT CLOSE AT END
\makeatletter
\let\oldr@@t\r@@t
\def\r@@t#1#2{%
\setbox0=\hbox{$\oldr@@t#1{#2\,}$}\dimen0=\ht0
\advance\dimen0-0.2\ht0
\setbox2=\hbox{\vrule height\ht0 depth -\dimen0}%
{\box0\lower0.4pt\box2}}
\LetLtxMacro{\oldsqrt}{\sqrt}
\renewcommand*{\sqrt}[2][\ ]{\oldsqrt[#1]{#2}}
\makeatother

\pagestyle{plain}

%==============================================================================
% General macros
%==============================================================================
% \renewcommand{\L}{\left}
% \newcommand{\R}{\right}
\renewcommand{\(}{\left(}
\renewcommand{\)}{\right)}
\renewcommand{\[}{\left[}
\renewcommand{\]}{\right]}

\newcommand{\T}{\top} % transpose symbol
\newcommand{\norm}[1]{\lVert #1 \rVert}
\newcommand{\code}[1]{\texttt{#1}}
\newcommand{\vect}[1]{\mathbf{#1}}

% fractions
\newcommand{\half}{\frac{1}{2}}
\newcommand{\quarter}{\frac{1}{4}}

% Greek letters
\newcommand{\z}{\zeta}
\newcommand{\sig}{\sigma}
\newcommand{\wn}{\omega_n}
\newcommand{\w}{\omega}
\newcommand{\wwd}{\omega_d}
\renewcommand{\th}{\theta}
\newcommand{\thd}{\dot{\theta}}
\newcommand{\thdd}{\ddot{\theta}}

% Trig functions
\newcommand{\ccp}{\cos\phi}
\newcommand{\ssp}{\sin\phi}
\newcommand{\cwt}{\cos\w t}
\newcommand{\swt}{\sin\w t}
\newcommand{\cwtp}{\cos(\w t - \phi)}
\newcommand{\swtp}{\sin(\w t - \phi)}

% unit vectors:
\newcommand{\er}{\textbf{e}_\textbf{r}}
\newcommand{\et}{\textbf{e}_\mathbf{\theta}}
\newcommand{\ex}{\textbf{e}_\textbf{x}}
\newcommand{\ey}{\textbf{e}_\textbf{y}}
\newcommand{\ez}{\textbf{e}_\textbf{z}}

\newcommand{\chat}{\mathbf{\hat{c}}}
\newcommand{\ihat}{\mathbf{\hat{i}}}
\newcommand{\jhat}{\mathbf{\hat{j}}}
\newcommand{\khat}{\mathbf{\hat{k}}}
\newcommand{\nhat}{\mathbf{\hat{n}}}
\newcommand{\phat}{\mathbf{\hat{\phi}}}
\newcommand{\shat}{\mathbf{\hat{s}}}
\newcommand{\that}{\mathbf{\hat{t}}}
\newcommand{\xhat}{\mathbf{\hat{x}}}
\newcommand{\yhat}{\mathbf{\hat{y}}}
\newcommand{\zhat}{\mathbf{\hat{z}}}

\newcommand{\dt}{\Delta t}

% Misc.
\newcommand{\Matlab}{\textsc{Matlab}}
\newcommand{\ie}{{\it i.e.\ }}
\newcommand{\eg}{{\it e.g.\ }}
\renewcommand{\deg}{^\circ}
\newcommand{\C}{\mathscr{C}}
\renewcommand{\L}{\mathscr{L}}

% Use for fractions with large terms
\newcommand{\ddfrac}[2]{\frac{\displaystyle #1}{\displaystyle #2}}

% Derivatives
\newcommand{\dd}{\mathrm{d}}
\newcommand{\ppx}[2]{\frac{\partial #1}{\partial #2}}
\newcommand{\psx}[2]{\frac{\partial^2 #1}{\partial #2^2}}
\newcommand{\ddx}[2]{\frac{\dd #1}{\dd #2}}
\newcommand{\dsx}[2]{\frac{\dd^2 #1}{\dd #2^2}}
\newcommand{\DDx}[2]{\frac{\mathrm{\D}#1}{\mathrm{\D} #2}}
\newcommand{\dx}[1]{\,\dd #1} % for inside integration

\newcommand{\xd}{\dot{x}}
\newcommand{\xdd}{\ddot{x}}
\newcommand{\yd}{\dot{y}}
\newcommand{\ydd}{\ddot{y}}
\newcommand{\zd}{\dot{z}}
\newcommand{\zdd}{\ddot{z}}
\newcommand{\qd}{\dot{q}}
\newcommand{\qdd}{\ddot{q}}
\newcommand{\wwdd}{\ddot{w}}

% Circled numbers
\newcommand{\circled}[1]{\raisebox{.5pt}{\textcircled{\raisebox{-.9pt} {#1}}}}


%==============================================================================
%     File-specific Macros
%==============================================================================
% Eliminate section numbers -- make subsections letters
\renewcommand{\thesection}{}
\renewcommand{\thesubsection}{(\alph{subsection})}
\renewcommand{\thesubsubsection}{}
% \numberwithin{equation}{section}	% Eqn numbers (chapter.section.number)

